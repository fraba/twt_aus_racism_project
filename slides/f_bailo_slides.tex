\documentclass[serif, aspectratio=169]{beamer}

\usetheme{metropolis}

% Define title
\newcommand{\doctitle}{Racism in online interactions}
\newcommand{\docsubtitle}{}% Remove if not needed

% Define author
\newcommand{\docauthor}{Francesco Bailo}
% \newcommand{\docauthortitle}{PhD Student}
\newcommand{\docauthorinstitute}{Department of Media and Communications, The University of Sydney}
\newcommand{\docauthoremail}{francesco.bailo@sydney.edu.au}
\newcommand{\InstitutePlace}{Gale Digital Scholar Lab}

% Define subject and keywords
\newcommand{\docsubject}{ADD SUBJECT HERE}
\newcommand{\dockeywords}{Keywords;}

% Define creation date (format D:YYYYMMDDHHmmss)
\newcommand{\doccreationdate}{D:201809031900}

\input{/Users/francesco/tex_inputs/biblatex_apa}
\input{/Users/francesco/tex_inputs/default_slides_set}


% Beamer buttons

\setbeamertemplate{button}{\tikz
  \node[
  inner xsep=4pt,
  draw=structure!80,
  fill=structure!50,
  rounded corners=4pt]  {\tiny\insertbuttontext};}

%----------------------------------------------------------------------------------------
%	TITLE SECTION
%----------------------------------------------------------------------------------------

\title{\doctitle \docsubtitle} % Article title

\author{
\large
\textsc{Francesco Bailo (with Gerard Goggin)}\\[2mm] 
\normalsize Department of Media and Communications, The University of Sydney \\
\InstitutePlace
% \normalsize \href{mailto:\docauthoremail}{\docauthoremail} 
% \vspace{-5mm}
}

\date{3 October 2018}

\usepackage[titletoc]{appendix}
\usepackage{graphicx}
\usepackage{media9}

%----------------------------------------------------------------------------------------


%----------------------------------------------------------------------------------------
%	HEADER AND FOOTER SECTION
%----------------------------------------------------------------------------------------

\setbeamertemplate{headline}{%
\leavevmode%
  \hbox{%
    \begin{beamercolorbox}[wd=\paperwidth,ht=2.5ex,dp=1.125ex]{palette quaternary}%
    \insertsectionnavigationhorizontal{\paperwidth}{}{\hskip0pt plus1filll}
    \end{beamercolorbox}%
  }
}


\setbeamertemplate{footline}[text line]{%
  \parbox{\linewidth}{\vspace*{-8pt} @FrBailo \hfill\insertpagenumber/\insertpresentationendpage
}}

%----------------------------------------------------------------------------------------

\newcommand{\customcite}[1]{\citeauthor{#1}, \citetitle{#1}, \citeyear{#1}}


\usepackage{newverbs}
\newverbcommand{\bverb}
  {\begin{lrbox}{\verbbox}}
  {\end{lrbox}\colorbox{gray!30}{\box\verbbox}}

\begin{document}

{
\setbeamertemplate{headline}{} 
\setbeamertemplate{footline}{} 
\begin{frame}
  \titlepage
\end{frame}
}
\addtocounter{framenumber}{-1}

\frame{\tableofcontents}

\section{Justification and research question}

\begin{frame}

According to a survey we conducted within the Digital Rights and Governance project among 1,600 Australians, about one-third of respondents who declare to speak a language other than English at home have been subjected to racism in their online interactions. \autocite{goggin_digital_2017}

\begin{figure}
\includegraphics{"/Users/francesco/public_git/twt_aus_racism_project/replication_package/survey_res_racism1"}
\end{figure}

\end{frame}

\begin{frame}
{Research question?}

Based on empirical observation, how likely an Australian Twitter user with a non-Anglo-Saxon background is subjected to racism?

\end{frame}


\section{Research design}

\begin{frame}
{Main challenges}

In order to answer to the research question, I need to

\begin{enumerate}

\item Create a \textit{representative sample} of Twitter users and tweets;

\item Identify the \textit{ethnic background} of Twitter users;

\item Assess if tweets are \textit{racist} or not.

\end{enumerate}

Note: The research project is under revision by the research and ethics committee of the University.

\end{frame}

\begin{frame}
{Creating a representative sample of Twitter users and tweets}

\begin{figure}
\includegraphics{"/Users/francesco/public_git/twt_aus_racism_project/replication_package/survey_res_racism2"}
\end{figure}

Twitter is used by less than 15\% of Internet users who speaks a language other than English but they are highly concentrated in cities. 


\end{frame}

\begin{frame}
{Distribution of Italian speaking population (Sydney, 2016 census)}

\begin{figure}
\includegraphics[width=0.75\textwidth]{"/Users/francesco/public_git/twt_aus_racism_project/replication_package/map_italian_speaking"}
\end{figure}

\end{frame}

\begin{frame}
{Distribution of Cantonese speaking population (Sydney, 2016 census)}

\begin{figure}
\includegraphics[width=0.75\textwidth]{"/Users/francesco/public_git/twt_aus_racism_project/replication_package/map_cantonese_speaking"}
\end{figure}

\end{frame}

\begin{frame}
{Distribution of Vietnamese speaking population (Sydney, 2016 census)}

\begin{figure}
\includegraphics[width=0.75\textwidth]{"/Users/francesco/public_git/twt_aus_racism_project/replication_package/map_vietnamese_speaking"}
\end{figure}

\end{frame}

\begin{frame}
{Distribution of Arabic speaking population (Sydney, 2016 census)}

\begin{figure}
\includegraphics[width=0.75\textwidth]{"/Users/francesco/public_git/twt_aus_racism_project/replication_package/map_arab_speaking"}
\end{figure}

\end{frame}

\begin{frame}
{Creating a representative sample of Twitter users and tweets}

\begin{itemize}
\item 23,401,892 (100\%) people living in Australia on census night 2016;
\item 6,388,717 (27\%) people speaking a language other than English (PLOE);
\item 4,400,000 (70\%, estim.) PLOE on the Internet;
\item 650,000 (14\%) PLOE on Twitter.
\end{itemize}

Residents speaking a language other than English tend to cluster in specific areas.

I can use the filtering functions of the Twitter API to collect tweets published in those areas.  

\end{frame}

\begin{frame}[fragile]
{Distribution of Arabic speaking population (Sydney, 2016 census)}

\begin{columns}
\begin{column}{0.7\textwidth}
\begin{figure}
\includegraphics[width=0.85\textwidth]{"/Users/francesco/public_git/twt_aus_racism_project/replication_package/map_arab_speaking_zoom"}
\end{figure}
\end{column}
\begin{column}{0.3\textwidth}
{\scriptsize Limits to filter realtime Tweets via API for \textit{Greenacre} \\ \tt{statuses/filter} \& \tt{locations=151.04014, -33.91810, 151.07729, -33.88318}}
\end{column}
\end{columns}

\end{frame}

\begin{frame}
{Testing Greenacre API query}

Over 4 days (long weekend):

\begin{itemize}

\item 35,584 tweets;
\item 41\% of tweets are a reply;
\item 6,338 Twitter users.

\end{itemize}

\end{frame}

\begin{frame}
{Creating a representative sample of Twitter users and tweets}

Defining 30 bounding boxes in Sydney, Melbourne, Brisbane and Adelaide and filtering results from the Streaming API based on the diffusion of languages.

\begin{itemize}

\item Italian;
\item Greek;
\item Arabic;
\item Cantonese;
\item Mandarin;
\item Vietnamese.

\end{itemize}

\end{frame}

\begin{frame}
{Identifying the ethnic background of Twitter users}

{\small

The identification of the ethnic background of Twitter users is conducted by comparing user names to a dictionary of common first names derived from \url{en.wiktionary.org}.

The dictionary is compiled by parsing all given names/surnames and their variations in the categories:

}

{\tiny

\begin{itemize}

\item Italian male given names;
\item Italian female given names;
\item Greek male given names;
\item Greek female given names;
\item ... 
\item English surnames from Chinese;
\item Vietnamese male given names‎.
\item Vietnamese female given names‎.

\end{itemize}

}

\end{frame}

\begin{frame}

\begin{figure}
\includegraphics[width=.9\textwidth]{"/Users/francesco/Desktop/wiktionary"}
\end{figure}

\end{frame}

\begin{frame}
{Identifying the ethnic background of Twitter users}

Labelling approaches:

\begin{itemize}

\item Simple match (given name/surname is present in Twitter user name);
\item Supervised learning: training an algorithm by manually coding a selected sample.

\end{itemize}

Issues:

\begin{itemize}

\item False positives (e.g. the Arabic name \texttt{Ali} will match \texttt{Ali} short for \textit{Alice});
\item Supervised learning requires significant time investment for manual labelling and assessment of predictions. 

\end{itemize}

\end{frame}

\begin{frame}
{Creating a representative sample of Twitter users and tweets}

When user names are identified as belonging to PLOE, replies directed to them will be pulled from TrISMA (trisma.org).

TrISMA have been tracking and storing tweets of the \enquote{most active Australian Twitter accounts}.

TrISMA data will allow to expand the dataset longitudinally. 

\end{frame}

\begin{frame}
{Assessing if tweets are racist or not}

Labelling approaches:

\begin{description}

\item[Dictionary-based approach]: A dictionary was created from the Wiktionary category \textttt{English ethnic slurs} (containing 332 terms).

\item[Supervised learning approach] based on three categories \textit{Toxicity}, \textit{Aggression} and \textit{Personal Attacks} based on the manually coded dataset released by the \enquote{Wikipedia Detox project}.

\end{description}


\end{frame}

\begin{frame}
{Assessing if tweets are racist or not}

The \textbf{Dictionary-based approach} will return a binary answer to the question if a tweet contains racist terms. 

But the \textbf{Supervised learning approach} will return a more nuanced answer based (racism is a subjective experience!) on three dimensions: 

\begin{description}

\item[Toxicity] How strongly the message is \enquote{perceived as likely to make [the recipient] want to leave the discussion}.

\item[Aggression] \enquote{How strongly the message is perceived as aggressive} towards the recipient.

\item[Personal Attack] \enquote{[W]hether it contains a personal attack} towards the recipient.

\end{description}

\end{frame}


\begin{frame}[allowframebreaks]
\printbibliography
\end{frame}



\end{document}
%%% Local Variables:
%%% mode: latex
%%% TeX-master: t
%%% End:
